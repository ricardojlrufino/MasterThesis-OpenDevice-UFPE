
\chapter{Conclusões}

A criação de projetos para Internet das Coisas lida com vários desafios,
na sua essência, relacionados à heterogeneidade dos dispositivos,
protocolos e tecnologias de comunicação. As arquiteturas baseadas
em middleware têm sido apresentadas como potenciais soluções para
lidar com estes desafios, conforme observamos no capítulo de trabalhos
relacionados. Porém, devido à grade variedade de domínios que estão
inseridos os projetos de IoT, dificilmente um único padrão de middleware
genérico irá resolver adequadamente os desafios de todos os domínios. 

Desta forma, este trabalho propõe uma solução baseada em framework,
com uma infraestrutura sólida de comunicação, abstração de dispositivos
e gerenciamento de armazenamento, permitindo o desenvolvimento de
soluções específicas para cada domínio.

A implementação do middleware permitiu avaliar que o mesmo pode ser
adotado em projetos menos complexos, mostrando-se um alternativa eficaz
para abstração de dispositivos e protocolos, simplificando o processo
de desenvolvimento de soluções de IoT.

A disponibilização de bibliotecas que auxiliam na construção de aplicações
embarcadas (firmware), permitem a simplificação do processo de desenvolvimento
de dispositivos inteligentes, habilitados para Internet das Coisas.
As ferramentas oferecidas neste trabalho, bem como o protocolo proposto,
permitem a utilização de microcontroladores de baixo custo e plataformas
de prototipação, como o Arduino, para criação destes dispositivos.

A disponibilização de bibliotecas para construção de aplicações cliente,
que implementam o protocolo proposto, permitem a integração de aplicações
usando linguagens de alto nível e simplificando o processo de desenvolvimento. 

Os experimentos realizados na seção \ref{sec:Integracao3D}, permitiram
avaliar a simplicidade da implementação do protocolo em outra linguagem
de programação.

Por fim, a arquitetura proposta, seu design e implementação, conseguiram
atender a todos objetivos estabelecidos no Capítulo 1, seção \ref{sec:Objetivos}. 


\section{Contribuições}

As principais contribuições deste trabalho são:
\begin{itemize}
\item Um framework para construção de projetos de Internet das Coisas, com
uma estrutura flexível e modular, baseado em padrões abertos.
\item Um framework de conexões, que permite a integração com novos servidores
e a inclusão de novos protocolos de comunicação com os dispositivos.
\item Um middleware genérico, multiplataforma e extensível, que permite
a abstração da comunicação com dispositivos heterogêneos, utilizando
tecnologias de comunicação USB, Bluetooth, Ethernet, Wi-Fi.
\item Uma interface Web para visualização e análise de dados que permite
a construção de dashboards dinâmicos, utilizando um série de gráficos
com suporte a visualização de dados históricos ou em tempo real.
\item Biblioteca JavaScript para criação de projetos Web, que realiza a
abstração dos dispositivos e permite comunicação em tempo real utilizando
WebSocket.
\item Um componente que permite a execução de aplicações em JavaScript no
lado do servidor ou como aplicações Desktop.
\item Biblioteca para integração com dispositivos móveis (Android).
\item Compilação e distribuição da biblioteca BlueCove (Bluetooth), para
plataforma ARM.
\item Experimentos com a API \emph{Device I/O}, que validaram sua compatibilidade
com a plataforma de desenvolvimento ``Beaglebone Black''.
\item Experimentos que oferecem os fundamentos para criação de sistemas
de simulação usando ambientes 3D.
\item A proposta de um protocolo aberto, simples, extensível e de fácil
implementação, que permite a comunicação com dispositivos com baixo
pode ser processamento e memória.
\item Um conjuntos de bibliotecas (firmware) em C++ para microcontroladores
e plataformas abertas (ex.: Arduino), que implementam o protocolo
proposto e simplificam o processo de desenvolvimento de aplicações
embarcadas. 
\end{itemize}

\section{Trabalhos Futuros}
\begin{itemize}
\item Desenvolvimento de uma plataforma como serviço (PaaS), que ofereça
uma infraestrutura escalável baseada em \emph{cloud computing}, para
desenvolvimento e implantação de projetos de IoT oferecidos como serviço
(SaaS). Essa plataforma seria desenvolvida com base no OpenStack\footnote{https://www.openstack.org/}
e/ou OpenShift\footnote{www.openshift.com}.
\item Criação de ferramentas para geração do firmware de forma automática
usando linguagens declarativas ou visuais.
\item Criação de ferramentas que permitam a atualização remota do firmware.
\item Otimizar a estrutura de armazenamento de dados do Neo4J, utilizando
um modelo baseado em arvore (GraphAware Neo4j TimeTree\footnote{http://graphaware.com/neo4j/2014/08/20/graphaware-neo4j-timetree.html}).
Este modelo pode oferecer uma melhor performance para trabalhar com
consulta e análise de eventos baseados no tempo.
\item Realizar estudos de caso em outros cenários, como casa inteligente,
cidade inteligente, redes de sensores sem fio, etc.
\item Implementação de algoritmos inteligentes para detecção de eventos
complexos, usando \emph{Complex Event Processing.}
\end{itemize}

