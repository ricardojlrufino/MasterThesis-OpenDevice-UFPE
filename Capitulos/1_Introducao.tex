
\chapter{Introdução}

A Internet das Coisas ou Internet of Things (IoT)\cite{Ashton2009},
em inglês, é um termo utilizado para descrever um paradigma tecnológico
no qual os objetos físicos estão conectados em rede e se comunicam
através da Internet, gerando informações, atuando de forma inteligente
a eventos, integrando aplicações e serviços.

A IoT é umas das principais tecnologias emergentes, é considerada
a terceira onda de desenvolvimento da Internet\cite{GoldmanSachs2014},
e assim como as duas primeiras ondas, promete trazer profundas mudanças
na economia. Essa ``revolução'' vem impulsionando mudanças nos padrões
da própria Internet, a fim de atender uma escala nunca antes vista
de ``coisas'' conectadas, e inevitavelmente, promovendo para vida
cotidiana, trabalho, saúde e negócios.

Recentes avanços tecnológicos tem impulsionado o desenvolvimento da
Internet das Coisas, a citar: (1) redução dos custos de sensores e
microprocessadores, permitindo a criação de dispositivos ``mais inteligentes'',
(2) redução no custo de banda da internet, algo estimado em 40x nos
últimos 10 anos, (3) a presença massiva de smartphones, tornando-se
uma ferramenta para geração de dados, visualização e controle, (4)
BigData, com a capacidade de processar enorme quantidades de dados
e gerar informações estratégicas para os negócios e (5) a alta disponibilidade
de acesso a rede internet de maneira quase ubíqua\cite{GoldmanSachs2014}.

O potencial estimado da Internet das Coisas tem despertado o interesse
e investimento de grandes empresas como Google, Microsoft, Intel,
Oracle, IBM, Cisco, Sansung. Não somente empresa, mas projetos e
fundações ligadas ao ``ecossistema'' open source, como Eclipse,
Arduino, Ubuntu/Linux, vem apostando na Internet das Coisas. 

Apesar dos benefícios e oportunidades, ainda existem uma série de
desafios que precisam ser superados para uma transição completa para
a Internet das Coisas\cite{mattern2010,teixeira2011}. Os desafios
estão relacionados à heterogeneidade, escalabilidade, interoperabilidade
, segurança, privacidade, grande volumes de dados, complexidade do
software e do hardware, que estão inseridos com graus de relevâncias
diferenciados nas áreas da saúde, transportes, telecomunicações, automação,
etc.

Um enorme esforço de investigação é necessário para fazer a IoT viável,
uma vez que muitas questões em aberto ainda persistem nesta área.
A comunidade científica está oferecendo várias tentativas para padronizar
totalmente o paradigma IoT. Além disso, o grande número de dispositivos
que são esperados integrar o ambiente de IoT, obriga um esquema de
endereçamento eficaz. Tais questões são geralmente discutidos na literatura\cite{yang2010,Katasonov2008}. 

\section{Motivação}

A Internet das Coisas é uma grande oportunidade para empresas e desenvolvedores
criarem novas soluções que conectem dispositivos à Internet. Novos
dispositivos, sensores, microcontroladores e plataformas de desenvolvimento
vêm surgindo, impulsionados por essas oportunidades e pelo desenvolvimento
da Internet das Coisas. Porém, essa grande quantidade de dispositivos
heterogêneos, decorrente da inerente diversidade de tecnologias de
hardware e software, se tornam um dos principais desafios para o desenvolvimento
de aplicações.

Devido à falta do estabelecimento de padrões das plataformas de hardware
e software, os desenvolvedores acabam criando suas próprias soluções,
o que contribui para o problema da interoperabilidade. Muitas das
soluções elaboradas, acabam tendo uma baixa qualidade, são fechadas
(proprietárias) ou devido a sua complexidade acabam atrasando ou mesmo
inviabilizando o atendimento dos principais requisitos, que é gerar
valor de mercado e promover a inovação.

Nesse contexto, plataformas de middleware são avaliadas como soluções
promissoras para prover tal interoperabilidade, realizar o gerenciamento
da crescente quantidade de dispositivos e integrar aplicações\cite{teixeira2011}.
Tais plataformas oferecem um meio padronizado para o acesso aos dados
e serviços através de uma interface de alto nível\cite{Bandyopadhyay2011},
com isso as aplicações não precisam lhe dar com os detalhes de baixo
nível relacionados às especificidades de cada dispositivo e padrões
de comunicação. A adoção de uma plataforma de middleware, pode contribuir
para facilitar o desenvolvimento de projetos de IoT.

Existem cerca de 9 milhões de desenvolvedores Java, o fornecimento
de frameworks e serviços \emph{open source} em plataforma Java irá
tornar mais fácil a adoção da linguagem para projetos de Internet
das Coisas e favorecer possíveis contribuições para o desenvolvimento
e aprimoramento da solução proposta nesse trabalho.

\subsection{Desafios}

Sistemas baseados em Internet das Coisas podem ser usado para diferentes
fins e áreas, de modo que, temos de enfrentar diferentes desafios.
Nesta seção, vamos explicar alguns dos desafios que precisam ser considerados
nas atividades de investigação:
\begin{itemize}
\item \textbf{Tecnologias de Borda:} Ao nível do hardware, são necessários
mais esforços de investigação para desenvolver a tecnologia de dispositivos
embarcados, sensores, atuadores e identificação (passiva e ativa),
uma vez que um sistema baseado na Internet das Coisas deve ser capaz
de reunir informações suficientes sobre o mundo real, empregando uma
grande variedade de dispositivos e ambientes. Assim, exige-se mais
esforços para conectar dispositivos heterogêneos e implantá-los em
aplicações da Internet das Coisas, e para fornecer suporte para novos
dispositivos.
\item \textbf{Tecnologias de Rede}: Em IoT, as coisas estão ligadas através
de diferentes tipos de redes, ou seja, rede móvel, com e sem fio.
Estas redes fornecem comunicação bidirecional em diferentes níveis
entre os objetos do mundo real, aplicações e serviços que são utilizados
pelas aplicações da Internet das Coisas. Esta estrutura altamente
distribuída deve fornecer interconexão com baixo consumo de energia,
enquanto os dados distribuídas podem causar problemas de privacidade.
\item \textbf{Middleware:} Em IoT, temos redes e dispositivos heterogêneos.
Sua heterogeneidade pode potencialmente aumentar com as novas tecnologias.
Para facilitar a utilização destes dispositivos por aplicações da
Internet das Coisas, devemos proteger a sua heterogeneidade. Portanto,
precisamos desenvolver um middleware seguro, escalável e semanticamente
enriquecido para lidar com a heterogeneidade dos dispositivos.
\item \textbf{Plataformas de Serviços:} Eles suportam uma gestão de alto
nível de todos os dispositivos envolvidos, de forma integrada, garantindo
a escalabilidade, alta disponibilidade, e a execução segura das funcionalidades
solicitadas a partir de dispositivos.
\end{itemize}

\section{Objetivo Geral\label{sec:Objetivos}}

Criação de um framework para facilitar o desenvolvimento de aplicações
para Internet das Coisa, com menor custo e tempo possível, abstraindo
dispositivos, protocolos e tecnologias de comunicação de comunicação,
bem como oferecer um conjunto de APIs para integração com outras aplicações.

\section{Objetivos Específicos}

Os objetivos específicos que este trabalho pretende alcançar, são
listador a seguir:
\begin{itemize}
\item \textbf{Solução completa para integração entre plataformas}: oferecendo
soluções para o desenvolvimento e integração de projetos nas plataformas
de Hardware, Desktop, Web e Mobile.
\item \textbf{Integração entre diferentes dispositivos e protocolos}: permitindo
integrar diferentes dispositivos e protocolos em um mesmo ambiente
de maneira transparente e consistente, focando em padrões abertos
e baseado em plataformas open source (hardware e software). 
\item \textbf{Utilização em dispositivos com limitações de processamento
e memória}: oferecer soluções para o desenvolvimento de dispositivos
embarcados com limitações de processamento e memória, como microcontroladores
(AVR 8-bits), mini PCs e dispositivos móveis.
\item \textbf{Oferecer um protocolo aberto e extensível}: Trata-se da elaboração
de um protocolo (baseado em comandos), para integrar aplicações e
dispositivos com limitações de memória e processamento, permitindo
uma comunicação bidirecional, em tempo-real e apoiada sobre padrões
abertos como MQTT, WebSocket, TCP, etc. Um dos principais objetivos
é que este protocolo seja de fácil implementação e extensível.
\item \textbf{Simplificação na criação de dispositivos inteligentes}: permitindo
que as empresas e indivíduos possam focar em questões de projeto,
design, marketing e comercialização dos novos dispositivos inteligente
que irão compor a Internet das Coisas.
\item \textbf{Oferecer uma plataforma base para criação de projetos especializados}:
através de uma arquitetura flexível, oferecer uma fundação para que
empresas e desenvolvedores possam estender a adaptá-la a regras de
casos de uso específicos das áreas de automação, transporte, cidades
inteligentes, etc.
\end{itemize}
Neste contexto é apresentado o OpenDevice, um framework que facilita
a criação de projetos de Internet das Coisas usando tecnologias de
baixo custo, como: Arduino\cite{url:arduino:intro}, Rasberry Pi\cite{url:raspberry},
ESP8266\cite{url:esp8266:espressif} e outros\cite{arduino-comp1,arduino-comp2,arduino-comp3}.

\section{Organização da Dissertação}

Este trabalho está organizado da seguinte forma. O capítulo 2 apresenta
os fundamentos para o desenvolvimento desta dissertação. O capítulo
3 apresenta e discute propostas e trabalhos relacionados com o desenvolvimento
de middlewares ou projetos correlacionados. O capítulo 4 apresenta
a arquitetura proposta e os seus componentes, assim como a implementação
dessa arquitetura. O capítulo 5 apresenta a avaliação e os testes
da implementação da proposta. Finalmente, o capítulo 6 apresenta as
conclusões da dissertação, assim como os trabalhos futuros. 
