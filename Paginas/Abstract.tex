The Internet of Things (IoT) is an area that has attracted much attention from academia as industry and demanded a complex architecture with various components that integrate hardware, sensors and actuators of various protocols and interfaces. The Internet of Things will change the way we behave and interact with the environment, creating new services in the areas of health, education, leisure, sports, automation, etc.
	
Being a relatively new area, it is also suffering from problems due to lack of standardization. Thus, the creation of IoT projects become challenging, with cost and high time or end up with a low quality because developers have to solve common problems, however complex, as the definition of protocols and device management, which slows development and the generation of innovative solutions.

This work proposes a complete and flexible framework that helps in the development of Internet solutions of things, such as home automation, sensor monitoring, robot control, energy monitoring and smart cities, covering all involved platforms: Desktop, Web Mobile, Hardware and offering services for device management, connections, data storage and visualization as well a proposed protocol for communication with low-cost hardware, based on the Arduino platform, ESP8266, Raspberry Pi and the like, can operate using the USB technologies, Bluetooth, Ethernet, WiFi and open protocols as MQTT and WebSocket.

Scenarios are implemented in the context of home automation / building and integration with 3D platforms to evaluate the proposed architecture and components, from a simple application to a more complex application. The experiments showed that the architecture components provide abstraction and extensibility needed to implement IoT projects in automation context, and that the architecture components has minimal performance impact 

\begin{keywords}
Internet of Things, IoT, Framework, Middleware, Arduino, MQTT, Embedded Systems, Home Automation.
\end{keywords}