A Internet das Coisas (IoT) é uma área que vem atraindo a atenção
tanto do meio acadêmico quanto da indústria e demanda uma complexa
arquitetura com diversos componentes, que integram hardwares, sensores
e atuadores dos mais variados protocolos e interfaces. A Internet
das Coisas irá mudar a forma como nos comportamos e interagimos com
o ambiente, gerando novos serviços nas áreas da saúde, educação, lazer,
esportes, automação, etc. 

Por ser uma área relativamente nova, vem sofrendo também com os problemas
devido à falta de padronização. Dessa forma, a criação de projetos
de IoT se tornam desafiadores, com custo e tempo elevados ou acabam
tendo uma baixa qualidade, pois os desenvolvedores têm que resolver
problemas comuns, porém complexos, como definição de protocolos e
gerenciamento de dispositivos, o que atrasa o desenvolvimento e a
geração de soluções inovadoras.

Nesta dissertação propõe-se um framework completo e flexível que auxilia
no desenvolvimento de soluções de Internet das Coisas, como: automação
residencial, monitoramento de sensores, controle de robôs, monitoramento
de energia e cidades inteligentes, abordando todas as plataformas
envolvidas: Desktop, Web, Mobile, Hardware e oferecendo serviços para
gerenciamento de dispositivos, conexões, armazenamento de dados e
visualização, bem como a proposta de um protocolo para a comunicação
com hardwares de baixo custo, baseados na plataforma Arduino, ESP8266,
Raspberry Pi e similares, podendo operar usando as tecnologias USB,
Bluetooth, Ethernet, WiFi e com protocolos abertos, como MQTT e WebSocket.

Foram implementados cenários no contexto de automação residencial/predial
e integração com plataformas 3D, para avaliar a proposta da arquitetura
e componentes, partindo de uma aplicação mais simples para uma aplicação
mais complexa. Os experimentos permitiram concluir que os componentes
da arquitetura fornecem a abstração e a extensibilidade necessária
para implementar projetos de IoT no contexto de automação, e que os
componentes da arquitetura impactam minimamente na performance do
projeto.

\begin{keywords}
Internet das Coisas, IoT, Framework, Middleware, Arduino, MQTT, Sistemas Embarcados, Automação Residencial.
\end{keywords}
